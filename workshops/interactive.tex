%! suppress = MissingImport
\RequirePackage{import}
\subimport{../common}{preamble}
\subimport{../common}{packages}
\subimport{../common}{vars}
\begin{document}

    \section{Kotlin}\label{sec:kotlin}
    \begin{frame}[c]
        \centering
        \Large
        \includesvg[scale=3]{../pictures/Kotlin-logo-2021}
        \linebreak
        \only<2->{\textit{DIY!}}
    \end{frame}

    \section{Was ist Kotlin?}\label{sec:was-ist-kotlin?}
\begin{frame}
    \slidehead
    \begin{itemize}[<+->]
    \item Eine moderne Programmiersprache (2011) erst nur für die JVM gedacht
    \item Entwickelt von JetBrains
    \item Scripting support
    \item \href{https://github.com/JetBrains/kotlin}{Open Source (Apache 2.0)}
    \item Objektorientiert, imperativ, funktional
    \item Statisch typisiert
    \item Kompiliert
    \item Interoperabel mit Java
    \item Interoperabel mit C/C++ (via \href{https://kotlinlang.org/docs/native-overview.html}{Kotlin/Native})
    \item Interoperabel mit JavaScript (via \href{https://kotlinlang.org/docs/js-overview.html}{Kotlin/JS})
    \item Bewertet deine FOP Hausübung
    \end{itemize}
\end{frame}


    \section{Kotlin im Einsatz}\label{sec:kotlin-im-einsatz}

    \subsection{Hello World}\label{subsec:hello-world}
    \begin{frame}
        \slidehead
        \inputCode[]{
            minted language=java,
            title=\codeBlockTitle{HelloWorld.java},
        }{\codeDir/java/HelloWorld.java}
        \inputCode[]{
            minted language=kotlin,
            title=\codeBlockTitle{hello-world.kts},
        }{\codeDir/kotlin/hello-world.kts}
    \end{frame}

    \subsection{Variablen}\label{subsec:variablen}
    \begin{frame}[c]
        \slidehead
        \begin{columns}[c]
            \begin{column}{.5\textwidth}
                \centering
                \Large
                \textbf{Impliziter Typ}
                \normalsize
                \inputCode[]{
                    minted language=kotlin,
                    title=\codeBlockTitle{variables-1a.kts},
                }{\codeDir/kotlin/variables-1a.kts}
            \end{column}
            \begin{column}{.5\textwidth}
                \only<2->{
                    \centering
                    \Large
                    \textbf{Expliziter Typ}
                    \normalsize
                    \inputCode[]{
                        minted language=kotlin,
                        title=\codeBlockTitle{variables-2a.kts},
                    }{\codeDir/kotlin/variables-2a.kts}
                }
            \end{column}
        \end{columns}
    \end{frame}

    \begin{frame}[c]
        \slidehead
        \centering
        \large
        \textbf{If-Statements}
        \vspace{1em}
        \normalsize
        \inputCode[]{
            minted language=kotlin,
            title=\codeBlockTitle{if-statements.kts},
        }{\codeDir/kotlin/if-statements.kts}
    \end{frame}

    \subsection{Funktionen}\label{subsec:functionen}
    \begin{frame}[c]
        \slidehead
        \centering
        \large
        \textbf{Basic Function}
        \vspace{1em}
        \normalsize
        \inputCode[]{
            minted language=kotlin,
            title=\codeBlockTitle{functions-1a.kts},
        }{\codeDir/kotlin/functions-1a.kts}
    \end{frame}

    \begin{frame}[c]
        \slidehead
        \centering
        \large
        \textbf{Basic Function - Unit}
        \vspace{1em}
        \normalsize
        \inputCode[]{
            minted language=kotlin,
            title=\codeBlockTitle{functions-1b.kts},
        }{\codeDir/kotlin/functions-1b.kts}
    \end{frame}

    \begin{frame}[c]
        \slidehead
        \centering
        \large
        \textbf{Nullability}
        \vspace{1em}
        \normalsize
        \inputCode[]{
            minted language=kotlin,
            title=\codeBlockTitle{nullability.kts},
        }{\codeDir/kotlin/nullability.kts}
    \end{frame}

    \section{Challenge 1}\label{sec:challenge-1}

    \begin{frame}
        \slidehead
        \Large
        \centering
        Erste Challenge
        \only<2->{
            \linebreak[2]
            \Huge
            Installation
        }
    \end{frame}

    \begin{frame}
        \slidehead
        \begin{itemize}[<+->]
            \item JetBrains ToolBox \url{jetbrains.com/toolbox}
            \item IntelliJ
            \item Kotlin Plugin
            \item Clone https://gitlab.ofahrt-d120.de/alex/kotlin-workshop-code
        \end{itemize}
    \end{frame}

    \section{Challenge 2 - Warm Up}\label{sec:challenge-2}
    \begin{frame}[c]
        \slidehead
        \begin{columns}[c]
            \begin{column}{.5\textwidth}
                \Large
                Fibonacci
                \only<2->{
                    \normalsize
                    \begin{itemize}
                        \item<2-> \texttt{fib(0) = 0}
                        \item<3-> \texttt{fib(1) = 1}
                        \item<4-> \texttt{fib(n) = fib(n-1) + fib(n-2)}
                    \end{itemize}
                }
            \end{column}
            \begin{column}{.5\textwidth}
                \centering
                \includesvg[scale=0.3]{../pictures/Fibonacci_Spiral}
            \end{column}
        \end{columns}
    \end{frame}

    \section{When Statement}\label{sec:when-statement}

    \subsection{No Subject}\label{subsec:no-subject}
    \begin{frame}[c]
        \slidehead
        \centering
        \large
        \textbf{When Statement without subject}
        \vspace{1em}
        \normalsize
        \inputCode[]{
            minted language=kotlin,
            title=\codeBlockTitle{when-without-subject.kts},
        }{\codeDir/kotlin/when/without-subject.kts}
    \end{frame}

    \subsection{With Subject}\label{subsec:with-subject}
    \begin{frame}[c]
        \slidehead
        \centering
        \large
        \textbf{When Statement with subject}
        \vspace{1em}
        \normalsize
        \inputCode[]{
            minted language=kotlin,
            title=\codeBlockTitle{when/with-subject-1.kts},
        }{\codeDir/kotlin/when/with-subject-1.kts}
    \end{frame}

    \begin{frame}[c]
        \slidehead
        \centering
        \large
        \textbf{Collapsing cases}
        \vspace{1em}
        \normalsize
        \inputCode[]{
            minted language=kotlin,
            title=\codeBlockTitle{when/with-subject-2.kts},
        }{\codeDir/kotlin/when/with-subject-2.kts}
    \end{frame}

    \begin{frame}[c]
        \slidehead
        \centering
        \large
        \textbf{Ranges}
        \vspace{1em}
        \normalsize
        \inputCode[]{
            minted language=kotlin,
            title=\codeBlockTitle{when/with-subject-3.kts},
        }{\codeDir/kotlin/when/with-subject-3.kts}
    \end{frame}

    \section{Function Syntax}\label{sec:function-syntax}

    \begin{frame}[c]
        \slidehead
        \centering
        \large
        \textbf{Block Body}
        \vspace{1em}
        \normalsize
        \inputCode[]{
            minted language=kotlin,
            title=\codeBlockTitle{functions-2.kts},
        }{\codeDir/kotlin/functions-2.kts}
    \end{frame}

    \begin{frame}[c]
        \slidehead
        \centering
        \large
        \textbf{Expression Body}
        \vspace{1em}
        \normalsize
        \inputCode[]{
            minted language=kotlin,
            title=\codeBlockTitle{functions-3.kts},
        }{\codeDir/kotlin/functions-3a.kts}
    \end{frame}

    \begin{frame}[c]
        \slidehead
        \centering
        \large
        \textbf{Expression Body with implicit return type}
        \vspace{1em}
        \normalsize
        \inputCode[]{
            minted language=kotlin,
            title=\codeBlockTitle{functions-3b.kts},
        }{\codeDir/kotlin/functions-3b.kts}
    \end{frame}

    \section{Fibonacci Recursive Solution}\label{sec:fibonacci-recursive-solution}
    \begin{frame}[c]
        \slidehead
        \centering
        \large
        \textbf{Fibonacci Recursive Solution}
        \vspace{1em}
        \normalsize
        \inputCode[]{
            minted language=kotlin,
            title=\codeBlockTitle{fibonacci-recursive-solution.kts},
        }{\codeDir/kotlin/fibonacci-recursive-solution.kts}
    \end{frame}

    \begin{frame}[c]
        \slidehead
        \centering
        \large
        \textbf{For-Loops}
        \vspace{1em}
        \normalsize
        \inputCode[]{
            minted language=kotlin,
            title=\codeBlockTitle{for-loops-1.kts},
        }{\codeDir/kotlin/for-loops-1.kts}
    \end{frame}

    \section{Challenge 3}\label{sec:challenge-3}
    \begin{frame}[c, fragile]
        \slidehead
        \begin{columns}[c]
            \begin{column}{.5\textwidth}
                \Large
                Rock Paper Scissors
                \vspace{1em}
                \normalsize

                \begin{columns}[c]
                    \begin{column}{.5\textwidth}
                        \begin{onlyenv}<2->
                            \mbox{}
                            \\
                            \textbf{Input:}
                            \begin{codeBlock}{
                                minted language=text,
                                title=Example Input,
                            }
                                PR
                                RR
                                SS
                                SR
                                PS
                            \end{codeBlock}
                        \end{onlyenv}
                    \end{column}
                    \begin{column}{.5\textwidth}
                        \begin{onlyenv}<3->
                            \mbox{}
                            \\
                            \textbf{Output:}
                            \begin{codeBlock}{
                                minted language=text,
                                title=Example Input,
                            }
                                P
                                R
                                S
                                R
                                S
                            \end{codeBlock}
                        \end{onlyenv}
                    \end{column}
                \end{columns}
            \end{column}
            \begin{column}{.5\textwidth}
                \centering
                \includesvg[scale=0.3]{../pictures/rock-paper-scissors}
            \end{column}
        \end{columns}
    \end{frame}

    \section{Deconstruction}\label{sec:deconstruction}

    \subsection{Variables}\label{subsec:variables}
    \begin{frame}[c]
        \slidehead
        \centering
        \large
        \textbf{Deconstructing a list}
        \vspace{1em}
        \normalsize
        \inputCode[]{
            minted language=kotlin,
            title=\codeBlockTitle{variables-from-list.kts},
        }{\codeDir/kotlin/deconstruction/variables-from-list.kts}
    \end{frame}
    \begin{frame}[c]
        \slidehead
        \centering
        \large
        \textbf{Deconstructing a string}
        \vspace{1em}
        \normalsize
        \inputCode[]{
            minted language=kotlin,
            title=\codeBlockTitle{variables-from-string.kts},
        }{\codeDir/kotlin/deconstruction/variables-from-string.kts}
    \end{frame}

    \subsection{Loops}\label{subsec:loops}
    \begin{frame}[c]
        \slidehead
        \centering
        \large
        \textbf{For-Loops with Destructuring Declaration}
        \vspace{1em}
        \normalsize
        \inputCode[]{
            minted language=kotlin,
            title=\codeBlockTitle{for-loops-2.kts},
        }{\codeDir/kotlin/for-loops-2.kts}
    \end{frame}

    \section{Extension Function}\label{sec:extension-function}

    \begin{frame}[c]
        \slidehead
        \centering
        \large
        \textbf{Extension Functions}
        \vspace{1em}
        \inputCode[]{
            minted language=kotlin,
            title=\codeBlockTitle{functions-4-parameter.kts},
        }{\codeDir/kotlin/functions-4-parameter.kts}
        \only<2->{
            \inputCode[]{
                minted language=kotlin,
                title=\codeBlockTitle{functions-4-receiver.kts},
            }{\codeDir/kotlin/functions-4-receiver.kts}
        }
    \end{frame}

    \begin{frame}[c]
        \slidehead
        \centering
        \large
        \textbf{Extension Functions Second Example}
        \vspace{1em}
        \normalsize
        \inputCode[]{
            minted language=kotlin,
            title=\codeBlockTitle{list-receiver-example.kts},
        }{\codeDir/kotlin/list-receiver-example.kts}
    \end{frame}

    \section{Challenge 3-continued}\label{sec:challenge-3-cont}
    \begin{frame}[c, fragile]
        \slidehead
        \begin{itemize}[<+->]
            \item Implement Challenge 4 using data class for game
            \item + Extension function for calculating result of game
        \end{itemize}

    \end{frame}

    \section{Challenge 4}\label{sec:challenge-5}
    \begin{frame}[c, fragile]
        \slidehead
        \begin{columns}[c]
            \begin{column}{.5\textwidth}
                \Large
                Rock Paper Scissors
                \vspace{1em}
                \normalsize

                \begin{columns}[c]
                    \begin{column}{.5\textwidth}
                        \begin{onlyenv}<2->
                            \mbox{}
                            \\
                            \textbf{Input:}
                            \begin{codeBlock}{
                                minted language=text,
                                title=Example Input,
                            }
                                PRPS
                                RRSR
                                SPPR
                                PRPS
                                PSPS
                            \end{codeBlock}
                        \end{onlyenv}
                    \end{column}
                    \begin{column}{.5\textwidth}
                        \begin{onlyenv}<3->
                            \mbox{}
                            \\
                            \textbf{Output:}
                            \begin{codeBlock}{
                                minted language=text,
                                title=Example Input,
                            }
                                S
                                R
                                S
                                R
                                S
                            \end{codeBlock}
                        \end{onlyenv}
                    \end{column}
                \end{columns}
            \end{column}
            \begin{column}{.5\textwidth}
                \centering
                \includesvg[scale=0.3]{../pictures/rock-paper-scissors}
            \end{column}
        \end{columns}
    \end{frame}
\end{document}
