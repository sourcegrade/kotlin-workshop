%! suppress = MissingImport
\RequirePackage{import}
\subimport{../common}{preamble}
\subimport{../common}{packages}
\subimport{../common}{vars}
\begin{document}
    \section{Kotlin}\label{sec:kotlin}
    \begin{frame}[c]
        \centering
        \Large
        \includesvg[scale=3]{../pictures/Kotlin-logo-2021}
        \\
        \pause
        \textit{Vorkurs}
        \pause
        \textit{: Part II}
    \end{frame}


    \section{Was steht heute auf dem Plan}\label{sec:plan}
    \begin{frame}[c]
        \slidehead
        \begin{itemize}
            \item IDEs
            \item \textbf{.kts} vs \textbf{.kt} Dateien
            \item Nullability (und Null-Safety)
            \item Klassen
            \item Lambdas
        \end{itemize}
    \end{frame}


    \section{IDE}\label{sec:ide}
    \begin{frame}[c]
        \slidehead
        \Large
        \centering
        Was ist eine IDE?
    \end{frame}

    \begin{frame}[c]
        \slidehead
        IDE steht für \textit{Integrated Development Environment}
        \pause
        \begin{itemize}
            [<+->]
            \item Es gibt viele!
            \begin{itemize}
                \item VSCode
                \item NetBeans
                \item Eclipse
                \item BlueJ
                \item \textbf{IntelliJ} (für heute relevant)
            \end{itemize}
        \end{itemize}
    \end{frame}

    \subsection{IntelliJ Installieren}\label{subsec:intellij-installieren}
    \begin{frame}[c]
        \slidehead
        \begin{itemize}
            [<+->]
            \item JetBrains Toolbox Herunterladen
            \item Student-License beantragen \textit{(Optional, aber empfohlen)}
            \item In Toolbox einloggen
            \item IntelliJ Community bzw.~Ultimate via Toolbox installieren
        \end{itemize}
        \onslide<5->
        \vspace{2em}
        \small
        \centering
        \url{https://wiki.tudagrade.org/preparation/installation-intellij/}
    \end{frame}


    \section{\textbf{.kts} vs \textbf{.kt} Dateien}\label{sec:kts-kt-dateien}
    \begin{frame}
        \slidehead
        \begin{columns}
            \begin{column}{0.5\textwidth}
                \begin{center}
                    \large
                    \textbf{.kts}
                \end{center}
                \begin{itemize}
                    \item<2-> \textit{Script} Dateien
                    \item<3-> Custom scripting context \textit{(z.B.~main, gradle)}
                    \item<4-> Für kleine Projekte gut geeignet
                    \item<5-> Direkter Einstiegspunkt (ohne Funktion)
                    \item<10-> File > New Project > \textbf{Empty} Project in IntelliJ
                \end{itemize}
            \end{column}
            \begin{column}{0.5\textwidth}
                \begin{center}
                    \large
                    \textbf{.kt}
                \end{center}
                \begin{itemize}
                    \item<6-> Normale \textit{Source} Dateien
                    \item<7-> Für größere Projekte gut geeignet
                    \item<8-> \textbf{fun main() { ... }} als Einstiegspunkt
                    \item<9-> Mit Build-Tools wie z.B.~Gradle
                    \item<10-> File > New Project > \textbf{New} Project
                \end{itemize}
            \end{column}
        \end{columns}
    \end{frame}

    \subsection{Dependencies in \textbf{.kts} Dateien}\label{subsec:dependencies-in-kts-dateien}
    \begin{frame}[c]
        \slidehead
        \begin{itemize}
            [<+->]
            \item Externe projekte können auch in \textbf{.main.kts} Dateien eingebunden werden
            \item Diese nennt man \textbf{Dependencies}
            \item Mit \textbf{@file:DependsOn("...")} können diese eingebunden werden
            \item Alle packages auf \url{https://central.sonatype.com/} verfügbar
            \item Beispiel: \textbf{@file:DependsOn("org.jline:jline:3.22.0")}
            \item Weitere Repositories können via \textbf{@file:Repository("...")} eingebunden werden
        \end{itemize}
    \end{frame}

    \livecoding

    \subsection{Dependencies in \textbf{.kt} Dateien}\label{subsec:dependencies-in-kt-dateien}
    \begin{frame}[c]
        \slidehead
        \begin{itemize}
            [<+->]
            \item Dependencies in normale Kotlin-Projekte einfügen sieht etwas anders aus
            \item Es sind aber die selben Dependencies verfügbar
            \item In der \textbf{build.gradle.kts} Datei können diese mit \textbf{implementation("...")} eingefügt werden
            \item Beispiel:
            \kotlinfile{../listings/gradle-dependency.gradle.kts}
        \end{itemize}
    \end{frame}

    \livecoding


    \section{Nullability}\label{sec:nullability}

    \begin{frame}
        \slidehead
        \large
        Was passiert wenn ein Wert nicht gesetzt ist?
        \pause
        \begin{onlyenv}<2>
            \kotlinfile{../listings/to-int-or-null.kts}
        \end{onlyenv}
        \begin{onlyenv}<3>
            \kotlinfile{../listings/to-int-or-null-fun.kts}
        \end{onlyenv}
        \begin{onlyenv}<4>
            \kotlinfile{../listings/to-int-or-null-fun-typed.kts}
        \end{onlyenv}
    \end{frame}

    \livecoding

    \begin{frame}
        \slidehead
        \begin{itemize}
            [<+->]
            \item In Kotlin gibt es keine\footnote{Wenn man sich an den Regeln hält} \textit{Null-Pointer-Exceptions}
            \item Es gibt \textbf{Nullable} und \textbf{Non-Nullable} Variablen
            \item \textbf{Nullable} Variablen können \textbf{null} sein
            \item \textbf{Non-Nullable} Variablen können \textbf{nicht null} sein
            \item \textbf{Nullable} Typen werden mit \textbf{?} gekennzeichnet
            \item \url{https://kotlinlang.org/docs/null-safety.html}
        \end{itemize}
    \end{frame}

    \subsection{Elvis-Operator ?:}\label{subsec:elvis-operator}
    \begin{frame}
        \slidehead
        \vspace{-1.5em}
        \begin{itemize}
            \item<1-> Mit dem Elvis-Operator \textbf{?:} können default Werte gesetzt werden
            \item<2-> Beispiel:
            \begin{onlyenv}<2>
                \kotlinfile{../listings/to-int-or-null-fun-typed-if.kts}
            \end{onlyenv}
            \begin{onlyenv}<3>
                \kotlinfile{../listings/to-int-or-null-fun-typed-elvis.kts}
            \end{onlyenv}
        \end{itemize}
    \end{frame}

    \subsection{Safe-Call-Operator ?.}\label{subsec:safe-call-operator}
    \begin{frame}
        \slidehead
        \vspace{-1.5em}
        \begin{itemize}
            \item<1-> Mit dem Safe-Call-Operator \textbf{?.} können Funktionen und Attribute auf \textbf{Nullable} Ausdrücke aufgerufen werden
            \item<2-> Beispiel:
            \begin{onlyenv}<2>
                \kotlinfile{../listings/string-length-nullable-java.kts}
            \end{onlyenv}
            \begin{onlyenv}<3>
                \kotlinfile{../listings/string-length-nullable.kts}
            \end{onlyenv}
        \end{itemize}
    \end{frame}

    \subsection{Elvis-Operator ?. Pt. 2}\label{subsec:elvis-operator-pt-2}
    \begin{frame}
        \slidehead
        \vspace{-1.5em}
        \begin{itemize}
            [<+->]
            \item Der Elvis-Operator akzeptiert auch \textbf{control flow statements}
            \begin{itemize}
                \item \textbf{return}
                \item \textbf{break}
                \item \textbf{continue}
            \end{itemize}
            \item Beispiel: \kotlinfile{../listings/string-length-nullable-return.kts}
        \end{itemize}
    \end{frame}

    \livecoding


    \section{Klassen}
    \begin{frame}[c]
        \slidehead
        \begin{itemize}
            [<+->]
            \item Klassen sind die Grundlage von \textbf{objektorientierte Programmierung}
            \item Klassen sind \enquote{baupläne} für \textbf{Objekte}
            \item Beispiel: \kotlinfile{../listings/class-pos.kts}
            \item Der \textbf{Konstruktor} ist eine Funktion, die ein \textbf{Objekt} dieser Klasse \textbf{Instanziiert}
            \item Beispiel: \kotlinfile{../listings/class-pos-constructor.kts}
        \end{itemize}
    \end{frame}

    \subsection{Klassen mit Funktionen}\label{subsec:klassen-mit-funktionen}
    \begin{frame}[c]
        \slidehead
        \begin{itemize}
            [<+->]
            \item Klassen können auch Funktionen enthalten
            \item Beispiel: \kotlinfile{../listings/class-pos-fun.kts}
        \end{itemize}
    \end{frame}


    \section{Exkurs: Funktionen}\label{sec:exkurs:-funktionen}
    \begin{frame}
        \slidehead
        \vspace{-1.5em}
        \begin{itemize}
            [<+->]
            \item Bis jetzt, haben wir Funktionen so geschrieben:
            \kotlinfile{../listings/fun-normal.kts}
            \item Diese können so aufgerufen werden:
            \kotlinfile{../listings/fun-normal-call.kts}
        \end{itemize}
    \end{frame}

    \subsection{Extension-Funktionen}\label{subsec:extension-funktionen}
    \begin{frame}
        \slidehead
        \vspace{-1.5em}
        \begin{itemize}
            [<+->]
            \item Eine \textbf{Extension-Funktion} ist eine Funktion, die auf einem Objekt aufgerufen werden kann
            \item Diese Funktion ist aber nicht in der Klasse definiert
            \item Ihr habt schon welche benutzt!
            \begin{itemize}
                \item \textbf{.toInt()}
                \item \textbf{.toDouble()}
            \end{itemize}
            \item Beispiel: \kotlinfile{../listings/ext-fun-syntax.kts}
        \end{itemize}
    \end{frame}

    \begin{frame}
        \slidehead
        \vspace{-1.5em}
        \begin{itemize}
            [<+->]
            \item Innerhalb einer Extension-Funktion kann auf das \textbf{Receiver-Objekt} mit \textbf{this} zugegriffen werden
            \item Beispiel: \kotlinfile[][top=0cm, bottom=0cm]{../listings/ext-fun-this.kts}
            \item \textbf{this} kann auch meistens weggelassen werden
            \item Beispiel: \kotlinfile[][top=0cm, bottom=0cm]{../listings/ext-fun-this-implicit.kts}
        \end{itemize}
    \end{frame}


    \section{Challenge Rock Paper Scissors}\label{sec:challenge-rps}
    \begin{frame}[c, fragile]
        \slidehead
        \begin{columns}[c]
            \begin{column}{.5\textwidth}
                \Large
                \centering
                Rock Paper Scissors\\
                \textcolor{red}{Challenge}
                \vspace{1em}
                \normalsize

                \begin{columns}[c]
                    \begin{column}{.5\textwidth}
                        \begin{onlyenv}<2->
                            \mbox{}
                            \\
                            \textbf{Input:}
                            \begin{codeBlock}{
                                minted language=text,
                                title=Example Input,
                            }
                                PRPS
                                RRSR
                                SPPR
                                PRPS
                                PSPS
                            \end{codeBlock}
                        \end{onlyenv}
                    \end{column}
                    \begin{column}{.5\textwidth}
                        \begin{onlyenv}<3->
                            \mbox{}
                            \\
                            \textbf{Output:}
                            \begin{codeBlock}{
                                minted language=text,
                                title=Example Input,
                            }
                                S
                                R
                                S
                                R
                                S
                            \end{codeBlock}
                        \end{onlyenv}
                    \end{column}
                \end{columns}
            \end{column}
            \begin{column}{.5\textwidth}
                \centering
                \includesvg[scale=0.3]{../pictures/rock-paper-scissors}
            \end{column}
        \end{columns}
    \end{frame}


    \section{Lambdas}\label{sec:lambdas}
    \begin{frame}[c]
        \slidehead
        \kotlinfile{../listings/for-each-question.kts}
    \end{frame}


    \begin{frame}[c]
        \slidehead
        \begin{itemize}
            [<+->]
            \item Lambdas sind \textbf{anonyme Funktionen}
            \item Sie können als Parameter übergeben werden
            \item Beispiel: \kotlinfile{../listings/for-each-lambda.kts}
        \end{itemize}
    \end{frame}

    \begin{frame}[c]
        \slidehead
        Funktionen können mit der Syntax \textbf{::funktion} als \textbf{Method Reference} implizit in ein Lambda umgewandelt werden
        \pause
        \kotlinfile{../listings/for-each-lambda-method-ref.kts}
    \end{frame}

    \begin{frame}[c]
        \slidehead
        Es ist auch möglich, statt Funktion, explizit ein Lambda-Ausdruck in einer Variable speichern
        \pause
        \kotlinfile{../listings/for-each-lambda-val.kts}
    \end{frame}

    \begin{frame}[c]
        \slidehead
        Der Lambda-Ausdruck kann auch direkt als Parameter übergeben werden
        \pause
        \kotlinfile{../listings/for-each-lambda-inline.kts}
    \end{frame}

    \begin{frame}[c]
        \slidehead
        Der Parameter \textbf{it} kann man auch weg lassen
        \pause
        \kotlinfile{../listings/for-each-lambda-inline-implicit-parameter.kts}
    \end{frame}

\end{document}
