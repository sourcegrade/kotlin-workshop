%! suppress = MissingImport
\RequirePackage{import}
\subimport{../common}{preamble}
\subimport{../common}{packages}
\subimport{../common}{vars}
\begin{document}

    \section{Kotlin}\label{sec:kotlin}
    \begin{frame}[c]
        \centering
        \Large
        \includesvg[scale=3]{../pictures/Kotlin-logo-2021.svg}
        \linebreak
        \only<2->{\textit{In 5 minutes}}
    \end{frame}


    \section{Was ist Kotlin?}\label{sec:was-ist-kotlin?}
    \begin{frame}
        \slidehead
        \begin{itemize}[<+->]
            \item Eine moderne Programmiersprache (2011) erst nur für die JVM gedacht
            \item Entwickelt von JetBrains
            \item Scripting support
            \item \href{https://github.com/JetBrains/kotlin}{Open Source (Apache 2.0)}
            \item Objektorientiert, imperativ, funktional
            \item Statisch typisiert
            \item Kompiliert
            \item Interoperabel mit Java
            \item Interoperabel mit C/C++ (via \href{https://kotlinlang.org/docs/native-overview.html}{Kotlin/Native})
            \item Interoperabel mit JavaScript (via \href{https://kotlinlang.org/docs/js-overview.html}{Kotlin/JS})
            \item Bewertet deine FOP Hausübung
        \end{itemize}
    \end{frame}


    \section{Kotlin im Einsatz}\label{sec:kotlin-im-einsatz}

    \subsection{Hello World}\label{subsec:hello-world}
    \begin{frame}
        \slidehead
        \inputCode[]{
            minted language=java,
            title=\codeBlockTitle{HelloWorld.java},
        }{\codeDir/java/HelloWorld.java}
        \inputCode[]{
            minted language=kotlin,
            title=\codeBlockTitle{hello-world.kts},
        }{\codeDir/kotlin/hello-world.kts}
    \end{frame}

    \subsection{Scripting}\label{subsec:scripting}
    \begin{frame}
        \slidehead
        \begin{columns}[c]
            \begin{column}{.5\textwidth}
                \centering
                \Large
                \textbf{*.kt}
                \vspace{1em}
                \normalsize
                \begin{itemize}
                    \item Normalerweise eine \textbf{Klassen-Datei}
                    \item Keine Anweisungen außerhalb von Klassen oder Funktionen
                \end{itemize}
                \only<2->{
                    \inputCode[]{
                        minted language=kotlin,
                        title=\codeBlockTitle{hello-world.kt},
                    }{\codeDir/kotlin/hello-world.kt}
                }
            \end{column}
            \begin{column}{.5\textwidth}
                \centering
                \Large
                \textbf{*.kts}
                \vspace{1em}
                \normalsize
                \begin{itemize}
                    \item \textbf{Skript-Datei}
                    \item Anweisungen ohne Klassen oder Funktionen
                \end{itemize}
                \only<2->{
                    \inputCode[]{
                        minted language=kotlin,
                        title=\codeBlockTitle{hello-world.kts},
                    }{\codeDir/kotlin/hello-world.kts}
                }
            \end{column}
        \end{columns}
    \end{frame}

    \subsection{Projekt Erstellung}\label{subsec:projekt-erstellung}
    \begin{frame}
        \slidehead
        \large
        \textbf{Wie kann man ein Kotlin-Projekt erstellen?}
        \begin{itemize}[<+->]
            \item Gradle init
            \item IntelliJ setup wizard
            \item Manuell
        \end{itemize}
        \only<4->{
            \vspace{2em}
            \centering
            \large
            \textbf{live-demo}
        }
    \end{frame}


    \section{Kotlin Syntax}\label{sec:kotlin-syntax}
    \begin{frame}[c]
        \centering
        \Large
        \textbf{Syntax}
    \end{frame}

    \subsection{Variablen}\label{subsec:variablen}
    \begin{frame}[c]
        \slidehead
        \begin{columns}[c]
            \begin{column}{.5\textwidth}
                \centering
                \Large
                \textbf{Impliziter Typ}
                \normalsize
                \inputCode[]{
                    minted language=kotlin,
                    title=\codeBlockTitle{variables-1a.kts},
                }{\codeDir/kotlin/variables-1a.kts}
            \end{column}
            \begin{column}{.5\textwidth}
                \only<2->{
                    \centering
                    \Large
                    \textbf{Expliziter Typ}
                    \normalsize
                    \inputCode[]{
                        minted language=kotlin,
                        title=\codeBlockTitle{variables-2a.kts},
                    }{\codeDir/kotlin/variables-2a.kts}
                }
            \end{column}
        \end{columns}
    \end{frame}

    \subsection{String Templates}\label{subsec:string-templates}
    \begin{frame}[c]
        \slidehead
        \begin{columns}[c]
            \begin{column}{.5\textwidth}
                \centering
                \Large
                \textbf{Impliziter Typ}
                \normalsize
                \inputCode[highlightlines={4}]{
                    minted language=kotlin,
                    title=\codeBlockTitle{variables-1a.kts},
                }{\codeDir/kotlin/variables-1b.kts}
            \end{column}
            \begin{column}{.5\textwidth}
                \centering
                \Large
                \textbf{Expliziter Typ}
                \normalsize
                \inputCode[highlightlines={4}]{
                    minted language=kotlin,
                    title=\codeBlockTitle{variables-2a.kts},
                }{\codeDir/kotlin/variables-2b.kts}
            \end{column}
        \end{columns}
    \end{frame}

%    TODO: Add comparison to Java

    \subsection{Funktionen}\label{subsec:functionen}
    \begin{frame}[c]
        \slidehead
        \centering
        \large
        \textbf{Basic Function}
        \vspace{1em}
        \normalsize
        \inputCode[]{
            minted language=kotlin,
            title=\codeBlockTitle{functions-1a.kts},
        }{\codeDir/kotlin/functions-1a.kts}
    \end{frame}

    \begin{frame}[c]
        \slidehead
        \centering
        \large
        \textbf{Basic Function - Unit}
        \vspace{1em}
        \normalsize
        \inputCode[]{
            minted language=kotlin,
            title=\codeBlockTitle{functions-1b.kts},
        }{\codeDir/kotlin/functions-1b.kts}
    \end{frame}

    \begin{frame}[c]
        \slidehead
        \centering
        \large
        \textbf{Block Body}
        \vspace{1em}
        \normalsize
        \inputCode[]{
            minted language=kotlin,
            title=\codeBlockTitle{functions-2.kts},
        }{\codeDir/kotlin/functions-2.kts}
    \end{frame}

    \begin{frame}[c]
        \slidehead
        \centering
        \large
        \textbf{Expression Body}
        \vspace{1em}
        \normalsize
        \inputCode[]{
            minted language=kotlin,
            title=\codeBlockTitle{functions-3.kts},
        }{\codeDir/kotlin/functions-3a.kts}
    \end{frame}

    \begin{frame}[c]
        \slidehead
        \centering
        \large
        \textbf{Expression Body with implicit return type}
        \vspace{1em}
        \normalsize
        \inputCode[]{
            minted language=kotlin,
            title=\codeBlockTitle{functions-3b.kts},
        }{\codeDir/kotlin/functions-3b.kts}
    \end{frame}

    \begin{frame}[c]
        \slidehead
        \centering
        \large
        \textbf{Extension Functions}
        \vspace{1em}
        \inputCode[]{
            minted language=kotlin,
            title=\codeBlockTitle{functions-4-parameter.kts},
        }{\codeDir/kotlin/functions-4-parameter.kts}
        \only<2->{
            \inputCode[]{
                minted language=kotlin,
                title=\codeBlockTitle{functions-4-receiver.kts},
            }{\codeDir/kotlin/functions-4-receiver.kts}
        }
    \end{frame}

    \begin{frame}[c]
        \slidehead
        \centering
        \large
        \textbf{Extension Functions Second Example}
        \vspace{1em}
        \normalsize
        \inputCode[]{
            minted language=kotlin,
            title=\codeBlockTitle{list-receiver-example.kts},
        }{\codeDir/kotlin/list-receiver-example.kts}
    \end{frame}

    \begin{frame}[c]
        \slidehead
        \centering
        \large
        \textbf{Nullability}
        \vspace{1em}
        \normalsize
        \inputCode[]{
            minted language=kotlin,
            title=\codeBlockTitle{nullability.kts},
        }{\codeDir/kotlin/nullability.kts}
    \end{frame}

    \begin{frame}[c]
        \slidehead
        \centering
        \large
        \textbf{If-Statements}
        \vspace{1em}
        \normalsize
        \inputCode[]{
            minted language=kotlin,
            title=\codeBlockTitle{if-statements.kts},
        }{\codeDir/kotlin/if-statements.kts}
    \end{frame}

    \begin{frame}[c]
        \slidehead
        \centering
        \large
        \textbf{For-Loops}
        \vspace{1em}
        \normalsize
        \inputCode[]{
            minted language=kotlin,
            title=\codeBlockTitle{for-loops-1.kts},
        }{\codeDir/kotlin/for-loops-1.kts}
    \end{frame}

    \begin{frame}[c]
        \slidehead
        \centering
        \large
        \textbf{For-Loops with Destructuring Declaration}
        \vspace{1em}
        \normalsize
        \inputCode[]{
            minted language=kotlin,
            title=\codeBlockTitle{for-loops-2.kts},
        }{\codeDir/kotlin/for-loops-2.kts}
    \end{frame}

    \begin{frame}[c]
        \slidehead
        \centering
        \large
        \textbf{While-Loops}
        \vspace{1em}
        \normalsize
        \inputCode[]{
            minted language=kotlin,
            title=\codeBlockTitle{while-loops.kts},
        }{\codeDir/kotlin/while-loops.kts}
    \end{frame}

    \begin{frame}[c]
        \slidehead
        \centering
        \large
        \textbf{Lists}
        \vspace{1em}
        \normalsize
        \inputCode[]{
            minted language=kotlin,
            title=\codeBlockTitle{lists.kts},
        }{\codeDir/kotlin/lists.kts}
    \end{frame}

\end{document}
